\section{Simulation}\label{sec:simulation}

A \emph{simulator} is a system that behaves \emph{similar} to another
system, but is implemented in a different way. The simulator provides
the the same --- or the subset of the same --- behaviour of the sytem
being simulated. As in, the simulator may or may not enforce all of
the rules imposed by the original system. This is in contrast to an
\emph{emulator} that exhibits the same \emph{exact} behaviour, as the
emulator completely replicates the system being
emulated,\footnote{Going so far as having binary compatible with the
  emulated system's inputs and outputs} albeit operating in a
different environment.

With respect to our simulator the goal is to execute a MIPS32 assembly
program and modifying the processor state as an effect of each
executed instruction. 

This means that the processor registers, program counter, data- and
instruction memory is all affected, together with all other
``hardware''. Our software parses the program in an
instruction-by-instruction basis and allows each instruction to alter
the processor state. Each instruction is decoded, and the associated
operation is subsequently performed;
\autoref{sec:implementation-requirements} outlines all of the
behaviours our simulation is expected to exhibit.

In the context of our single-cycle implementation our datapath
contains all the functional units specified in
\autoref{sec:simulation-requirements} with the additions of
some multiplexors.

