\subsubsection{Example: Numeric decoding}\label{sec:example-numeric-decoding}

Consider the machine-language MIPS32 instruction \tt{0x71014802}.
Depending on the format of the instruction it decomposes into fields
varying lengths.

Recall that for all numbers in the MIPS32 instruction set the leftmost
six bits always represent the opcode for the instruction. The opcode
alone is not always sufficient to identify the particular instruction,
\emph{but} it is always sufficient to identify the format of the
instruction.

The leftmost six bits of \tt{0x71014802} is \tt{0x1c}. It is
\emph{known} that this number corresponds to a set of instructions in
the R-type format. The format specifies into which fields the 32-bit
decomposes into. The number of bits composing each respective field is
given in the bottom row of ~\autoref{fig:r-decomposed},

\begin{figure}[H]
\centering
\begin{tikzpicture}[node distance=1em]

\matrix (decomposed-representation) [
  matrix of nodes,
  row sep=0.2em,
  column sep=-\pgflinewidth,
  row 1/.style={
    nodes={
      rectangle, 
      draw, 
      text centered,
          text width=10mm,
      anchor=base,
      text height=.8em,text depth=.2em,minimum size=7mm
    }
  }
] {
op & rs & rt & rd & shamt & funct \\
6 & 5 & 5 & 5 & 5 & 6 \\
};
\end{tikzpicture}

\caption{The length of each respective field for R-type format instructions}
\label{fig:r-decomposed}
\end{figure}

Decomposing \tt{0x71014802} into the fields shown in
\autoref{fig:r-decomposed} yields \tt{rs=8}, \tt{rt=1}, \tt{rd=9},
\tt{shamt=0}, and \tt{funct=2}. The \emph{decomposed representation}
of this instruction in hexadecimal form is thus \tt{[0x1c 8 1 9 0
  2]}.\footnote{The corresponding \emph{decimal representation} is
  \tt{[28 8 1 9 0 2]}}

To identify the particular instruction represented by \tt{0x71014802}
the \funct{} field must be consulted. Pairing the opcode, \tt{0x1c}
and the value in the \funct{} field uniquely identifies the
instruction a \tt{mul} instruction; see \autoref{fig:mul-decomposed}.

\begin{figure}[H]
  \centering
  \begin{tikzpicture}[node distance=1em]
\matrix (decomposed-representation) [
  matrix of nodes,
  row sep=0.2em,
  column sep=-\pgflinewidth,
  row 1/.style={
    nodes={
      rectangle, 
      draw, 
      text centered,
          text width=10mm,
      anchor=base,
      text height=.8em,text depth=.2em,minimum size=7mm
    }
  }
] {
0x1c & rs & rt & rd & 0 & 2 \\
6 & 5 & 5 & 5 & 5 & 6 \\
};
\node [left=of decomposed-representation] (mnemonic-representation) {\texttt{mul \$rd, \$rs, \$rt}};
\end{tikzpicture}

  \caption{Decomposition and mnemonic representation of \tt{mul}}
  \label{fig:mul-decomposed}
\end{figure}

Earlier we determined the fields \tt{rs}, \tt{rt} and \rd{} to have
the addresses 8, 1, and 9, respectively. In MIPS registers are named,
following the convention shown in
\autoref{table:mips-register-naming-and-usage-convention}

Replacing the numerical values of \tt{rs}, \tt{rt} and \rd{}, with
their named counterparts yields the \emph{mnemonic representation} of
the instruction to be

\begin{lstlisting}[style=mips_lst]
mul $t1, $t0, $at
\end{lstlisting}
%$
