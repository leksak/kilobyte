% The declaration of the document class:

% The second line here, i.e.
%
% \documentclass[a4paper]{article}  
%
% is a standard LaTeX document class declaration: 
% we say what kind of document we are making in curly brackets, 
% and specify any options in square brackets.
\documentclass[a4paper]{article}

\usepackage[T1]{fontenc}
\usepackage[utf8]{inputenc}
\usepackage[english]{babel}

% The following package allows us to define our own macros
% through "\LetLtxMacro\ourName\originalName"
\usepackage{letltxmacro}
\usepackage{csquotes}

% Adds vertical space between paragraphs, i.e. a clean linebreak
\usepackage{parskip} 

% Used for typesetting ordinary math
\usepackage{amsmath}

% These packages enables other useful math commands
\usepackage{amssymb}
\usepackage{amsthm}
\usepackage{amsfonts}

\usepackage{letltxmacro}
\LetLtxMacro\tt\texttt

\usepackage{graphicx}

% Allows us to use syntax highlighting on included files an
% in an inline fashion.
\usepackage{minted}

% Ignore syntax errors - the minted lexer is not perfect (MATLAB only)
\makeatletter
\expandafter\def\csname PYGdefault@tok@err\endcsname{\def\PYGdefault@bc##1{{\strut ##1}}}
\makeatother

% The package offers the command \DeclareFloatingEnvironment, which
% the user may use to define new floating environments which behave
% like the LaTeX standard foating environments figure and table.
\usepackage{newfloat}

% Improves the interface for defining floating objects such as figures
% and tables. You can define your own floats and improve the
% behaviour of the old ones.  
% 
% The package also provides the "H" float modifier option, which
% puts figures exactly where they are included as opposed to being
% placed where LaTeX deems as the most appropriate position
\usepackage{float} 
\floatplacement{figure}{H} % Set the H option globally

% Creates the Listing environment whereby you may specify fontsizes
% et al. when using \lstlisting
\usepackage{listings}
\usepackage{mips} % Load in the MIPS keywords

% Teach autoref how to reference listings
\providecommand*{\listingautorefname}{Listing} 

% Disallow the placement of the listing environment to be moved by LaTeX
\floatplacement{listing}{H}

\usepackage{xcolor}

\definecolor{keyword}{HTML}{A71D5D}
\definecolor{ident}{HTML}{333333}
\definecolor{background}{HTML}{F7F7F7}
\definecolor{registers}{HTML}{0086B3}

\lstset{
  aboveskip=20pt,                     % Vertical space above the listing
  belowskip=10pt,                     % Vertical space below the listing
  breaklines=true,                    % sets automatic line breaking
  basicstyle=\ttfamily,
  extendedchars=true,
  tabsize=2,
  columns=fixed,
  showstringspaces=false,
  captionpos=b,                       % sets the caption-position to bottom
  %
  % Insert a red arrow to highlight line-continuations
  postbreak=\raisebox{0ex}[0ex][0ex]{\ensuremath{\color{red}\hookrightarrow\space}},
}

\lstdefinestyle{mips_lst}{
  frame=trbl,
  language={[mips]Assembler},
  framesep=4pt,
  keywordstyle=\color{keyword},       % Keyword coloring
  identifierstyle=\lst@ifdisplaystyle\color{registers}\fi,
  backgroundcolor=\color{background},
}{}

\lstdefinestyle{semantics_lst}{
  aboveskip=2\medskipamount,
  belowskip=\medskipamount,
  mathescape=true, 
}{}

\usepackage[flushmargin]{footmisc} % Footnote position

\usepackage{hyperref}
\hypersetup{
  colorlinks = true, % Colours links instead of ugly boxes
  urlcolor = blue, % Colour for external hyperlinks
  linkcolor = blue, % Colour of internal links
  citecolor = red % Colour of citations
}

% Used to get the last page of the document, used in our document headers
\usepackage{lastpage}

% Specify date format
\usepackage[yyyymmdd]{datetime}
\renewcommand\dateseparator{-} % Separate date tokens with a "-"

\usepackage{booktabs}

% Use fancy document headers
\usepackage{fancyhdr}
\usepackage{tikz}

\usetikzlibrary{
  matrix,
  positioning
}

\newcommand{\shamt}{\texttt{shamt}}
\newcommand{\funct}{\texttt{funct}}
\newcommand{\opcode}{\texttt{opcode}}
\newcommand{\rd}{\texttt{rd}}
\newcommand{\rs}{\texttt{rs}}
\newcommand{\rt}{\texttt{rt}}
\newcommand{\PC}{\texttt{PC}}

\usepackage{microtype}

% An environment for nice looking quotes
\makeatletter
\newenvironment{chapquote}[2][2em]
  {\setlength{\@tempdima}{#1}%
    \def\chapquote@author{#2}%
    \parshape 1 \@tempdima \dimexpr\textwidth-2\@tempdima\relax}
  {\par\normalfont\hfill\ \chapquote@author\hspace*{\@tempdima}\par}
  \makeatother

%
% BibLaTeX
%
% Great reference system
% Documentation:
% http://mirrors.ctan.org/macros/latex/contrib/biblatex/doc/biblatex.pdf
%___________________________________________________________
\usepackage[style = ieee, urldate =comp, backend=bibtex]{biblatex}
\addbibresource{references.bib}

\renewcommand\dateseparator{-}

\newcommand{\department}{Department of Computing Science}
\newcommand{\university}{Ume\aa\ University}

\newcommand{\authors}{Filip Allberg (\tt{filip@cs.umu.se}) \\
                      Jonathan Westin (\tt{jwestin@cs.umu.se})}
\author{\authors}
\newcommand{\coursename}{Computer Organization and Architecture}
\newcommand{\coursecode}{5DV118}
\newcommand{\instructor}{Thomas Johansson}

\pagestyle{fancy}
% Header settings
\fancyhead[R]{\thepage(\pageref{LastPage})}
\fancyheadoffset[L,R]{12mm}
\fancyhead[L]{\coursename{}: \titlename{} \\ \authors{}}
\fancyfoot[L,R,C]{}

