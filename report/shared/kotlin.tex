\section{Kotlin}

Kotlin is a statically-typed programming language that runs on the
Java Virtual Machine (JVM). While not syntax compatible with Java,
Kotlin is designed to interoperate with Java code and is reliant on
Java code from the existing Java Class Library, such as the
collections framework.

\subsection{Why Kotlin}

Kotlin compiles to JVM bytecode (or JavaScript, but that is not
pertinent to this project). It solves problems commonly associated
with Java, in particular it requires far less boiler-plate to the
benefit of legibility without incurring a refactor-ability
cost.\footnote{Project Lombok can also offset a lot of Java
  boiler-plate, and it is a ``plain'' Java library}

Additionally its type system helps us avoid null-pointer exceptions
while still retaining a notion of null which is useful when working
with APIs that do (Java).

It does this by making a distinction between nullable and non-nullable
datatypes. All nullable objects must be declared with a ``\tt{?}''
postfix after the type name. Operations on nullable objects need
special care from developers; a null-check must be performed before
using the value. Kotlin provides special operators for this.

\subsection{Closing remarks}

Beyond the advantages mentioned above there are additional benefits
such as sum types, named arguments, functions that can be defined
outside of a class. All of these together has provided us with a
positive programming experience and allowed us to succinctly formulate
parts of our solution. Below is an example of how we define an
Instruction within the code-base.

\begin{figure}
\begin{lstlisting}[language=java, style=plain, caption=Kotlin --- applied]
@JvmField val ADD = Instruction(
  iname = "add",
  opcode = 0,
  funct = 32,
  mnemonicRepresentation = "add \$t1, \$t2, \$t3",
  numericRepresentation = 0x014b4820,
  description = "Addition with overflow,. Put the" +
    " sum of registers rs and rt into register" +
    " rd. Is only valid if shamt is 0.",
  format = Format.R,
  pattern = INAME_RD_RS_RT)
\end{lstlisting}
\end{figure}

Compare this with the corresponding definition shown in
Listing.~\ref{listing:MARS-instruction-definition} from the
Mars\cite{MARS-simulator} project,

\begin{lstlisting}[
language=java,
style=plain,
caption=MARS-simulator instruction definition,
label=listing:MARS-instruction-definition]
instructionList.add(
  new BasicInstruction("nop",
    "Null operation : machine code is all zeroes",
    BasicInstructionFormat.R_FORMAT,
    "000000 00000 00000 00000 00000 000000",
    new SimulationCode()
      {
         public void simulate(ProgramStatement statement) throws ProcessingException
           {
             // Hey I like this so far!
           }
      }));
\end{lstlisting}

which is subjectively more verbose than our presented counter-part.
