\section{Kotlin}

Kotlin is a statically-typed programming language that runs on the
Java Virtual Machine (JVM). While not syntax compatible with Java,
Kotlin is designed to interoperate with Java code and is reliant on
Java code from the existing Java Class Library, such as the
collections framework.

\subsection{Why Kotlin}

Kotlin compiles to JVM bytecode (or JavaScript, but that is not
pertinent to this project). It solves problems commonly associated
with Java, in particular it requires far less boiler-plate to the
benefit of legibility without incurring a refactor-ability
cost.\footnote{Project Lombok can also offset a lot of Java
  boiler-plate, and it is a ``plain'' Java library}

Additionally its type system helps us avoid null-pointer exceptions
while still retaining a notion of null which is useful when working
with APIs that do (Java).

It does this by making a distinction between nullable and non-nullable
datatypes. All nullable objects must be declared with a ``\tt{?}''
postfix after the type name. Operations on nullable objects need
special care from developers; a null-check must be performed before
using the value. Kotlin provides special operators for this.

\subsubsection{Sum types}

% TODO: Should we continue using io.atlassian.fugue.Either or go for
% a sum type with valid and partially valid instruction

\subsubsection{Singletons and class-less functions}

Want a singleton? Create an object:

object ThisIsASingleton {
    val companyName: String = "JetBrains"
  }

  \subsubsection{Named arguments}

\subsection{Kotlin --- applied}

\begin{lstlisting}[style=java]
@JvmField val ADD = Instruction(
      iname = "add",
      opcode = 0,
      funct = 0x20,
      mnemonicRepresentation = "add \$t1, \$t2, \$t3",
      numericRepresentation = 0x014b4820,
      description = "Addition with overflow,. Put the" +
            " sum of registers rs and rt into register" +
            " rd. Is only valid if shamt is 0.",
      format = Format.R,
      pattern = ParametrizedInstructionRoutine.INAME_RD_RS_RT)
\end{lstlisting}

