\section{Introduction}

In the MIPS32 architecture, all machine instructions are represented
as 32-bit numbers. This article presents a MIPS32-decompiler that,
when passed 32-bit numbers which either partially or completely
represent MIPS32 instructions yields a series of different
representations of the same instruction.

MIPS32 instructions are to be read from a file with numbers either in
decimal or hexadecimal form. For each number in the input file, the
disassembler will produce the following:

\begin{itemize}
  \item The number from the input file.
  \item The format of the instruction (R, I, or J).
  \item The decomposed representation in decimal.
  \item The decomposed representation in hexadecimal.
  \item The representation of the instruction in mnemonic format,
    using register abbreviations wherever possible (e.g.,
    \texttt{\$t0} instead of \texttt{\$8}) and using decimal numbers
    whenever actual numbers are necessary.
\end{itemize}

In the following subsections an introduction of the terminology used
throughout this document is provided. Afterwards you may refer to this
section again if any of the above requirements seem foreign to you.

The rest of the document will be dedicated to a high-level description
of this solution accompanied with a guide on how to compile and run
the software.

The appendix contains an overview of all those instructions that the
decompiler is capable of comprehending.

\subsection{Terminology}

According to Aho et al.\cite{Aho:2006:CPT:1177220} a \emph{compiler}
is a program that can read a program in one language --- the
\emph{source} language -- and translates it into an equivalent program
in another language -- the \emph{target} language; see
Fig.~\ref{fig:compiler}.

\begin{figure}[H]
  \centering
  \begin{tikzpicture}[>=stealth]
  % Draw the labelled box. ``compiler'' is its label, and how
  % it is referred to within this TikZ picture. Its displayed
  % text is ``Compiler''. I.e. generically we'd write
  %
  % \node [draw] (nodelabel) {some text};
  %
  \node [draw] (compiler) {Compiler};

  % Create an input node, with the text ``source program'' 
  % which is above the box named ``compiler'', see above.
  \node [above=of compiler] (input) {source program};

  % Create an output node, with the text ``target program'' 
  % which is below the box named ``compiler'', see above.
  \node [below=of compiler] (output) {target program};

  % Draw an arrow _from_ the input node, to the compiler node.
  \draw [->] (input) -- (compiler);

  % Draw an array _from_ the compiler node to the output node.
  \draw [->] (compiler) -- (output);
\end{tikzpicture}

  \caption{A compiler}
  \label{fig:compiler}
\end{figure}

Conversely, a \emph{decompiler} is also a that performs the reverse
operation of a compiler. It too, is a compiler. Commonly one views a
compiler as a translator from a high-level human-readable source
language into a low-level machine-readable language, similarly a
decompiler translates in the opposite direction; see
Fig. \ref{fig:decompiler}. 

The two exhibit a chiral relation to one another, i. e. that they are
mirrored images of each other in terms of functionality, but they are
not themselves identical.

\begin{figure}[H]
  \centering
  \begin{tikzpicture}[>=stealth]
  \node [draw] (decompiler) {Decompiler};
  \node [above=of decompiler, align=center] (input) {executable code \\ (machine legible)};
  \node [below=of decompiler, align=center] (output) {source code \\ (human legible)};
  \draw [->] (input) -- (decompiler);
  \draw [->] (decompiler) -- (output);
\end{tikzpicture}

  \caption{A decompiler}
  \label{fig:decompiler}
\end{figure}

In this assignment the decompiler must be able to translate from the
machine-readable language of 32-bit numbers representing MIPS32
instructions to the human-readable target language described in the
introduction.

