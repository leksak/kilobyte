\section{Usage and compilation}

We will now describe how to compile and use the supplied software, so
that you may experiment with it later when the solution is described.

The source code may be downloaded at the following two addresses,

\begin{center}
\url{https://github.com/leksak/kilobyte.git} \\
\url{git@github.com:leksak/kilobyte.git}
\end{center}

Or viewed in the following folder
\tt{\raise.17ex\hbox{$\scriptstyle\mathtt{\sim}$}filip/edu/kilobyte}
on the institution's network.

For the duration of this report the directory created by cloning
the repository is referred to as the \emph{root directory}.

\section{Compilation and usage}

In this section the compilation of the software is described as
is the usage of the software

\subsection{Compilation}

Navigate to the root directory of the project.

Whilst there execute the following command to compile the program:

\begin{lstlisting}[style=plain]
$ gradle fatJar
\end{lstlisting}

You will require Gradle 3.2.1, which is available for
\href{http://gradle.org/gradle-download/}{download here}, or
executable through
\tt{\raise.17ex\hbox{$\scriptstyle\mathtt{\sim}$}filip/bin/gradle}
on the institution's network.

In the \tt{build/libs} directory you will now find a file named
\tt{kilobyte-all-1.0-SNAPSHOT.jar} which you can run using \tt{java -jar
  kilobyte-all-1.0-SNAPSHOT.jar} given that you have Java 8 shown when running
\tt{java -version}.

\subsubsection{Compiling and running tests}

Navigate to the root directory and hilst there execute the following
command to compile and run the tests:

\begin{lstlisting}[style=plain]
$ gradle test
\end{lstlisting}

In Figure.~\ref{fig:tests} the output yielded by the tests is shown.

\begin{figure}[htpb]
\begin{lstlisting}[style=plain]
$ ~filip/bin/gradle test             
[...removed output...]
Test run finished after 258 ms
[        11 containers found      ]
[         0 containers skipped    ]
[        11 containers started    ]
[         0 containers aborted    ]
[        11 containers successful ]
[         0 containers failed     ]
[        26 tests found           ]
[         1 tests skipped         ]
[        25 tests started         ]
[         0 tests aborted         ]
[        25 tests successful      ]
[         0 tests failed          ]
[...removed output...]
\end{lstlisting}
\caption{Test results}
\label{fig:tests}
\end{figure}

\subsection{Usage}

The program allows us multiple means of interaction which are listed
when the program is passed the "-h" or "--help" flag which outputs the following
The program supplies two means of interfacing with it, by calling
the program with 0 arguments or the \tt{-h} flag we can get the
user guide supplied with the software

\begin{lstlisting}[style=plain]
$ usage: Decompiler [OPTION] [file|number]...
    --examples       prints an example for each supported instructions
 -h,--help           print this message
 -n,--number <arg>   disassemble 32-bit word(s) from stdin
 -S,--suppress       suppress the table header
    --supported      prints all supported instructions
\end{lstlisting}

\subsubsection{Decompiling source code from files}

The default setting is that the software interprets its given input
arguments as a filenames, relative to the directory from which you
execute the \tt{.jar} file. In the root directory of the project three
example files are included.

\tt{sample-program.hex} showcases the tabulated output for several
instructions, as well as how error messages for partially legal
instructions are written out.

\tt{sample-decimal.hex} showcases that the program is capable
of handling instructions in base 10. 

\tt{sample-mingled-bases.hex} demonstrates that the program does not
expect the file to be written consistently in either base 10 or base
16 but evaluates this on a per-line basis. A number is interpreted to
be in base 16 if it is prefixed by \tt{0x}. Between the two numbers
a blank-line appears to showcase that the decompiler is robust
enough to ignore such lines.

We can inspect the contents of these files using \tt{cat}
as shown in Listing.~\ref{listing:sample-input-files}.

\begin{figure}
\begin{lstlisting}[style=plain,
    caption=Sample input files,
    label=listing:sample-input-files]
$ cat sample-decimal.hex 
599654392

$ cat sample-mingled.hex
599654392

0xafbf0004

$ cat sample-program.hex
0x23bdfff8
0xafbf0004
0xafa40000
0x28880001
0x11000003
0x20020001
0x23bd0008
0x03e00008
0x2084ffff
0x0c100000
0x8fa40000
0x8fbf0004
0x23bd0008
0x70821002
0x03e00008
0x00012122
\end{lstlisting}
\end{figure}

Similarly we may observe the output of our decompiler in
Listing.~\ref{listing:tabularized-output}, where the
files are passed as a series of arguments.

\subsubsection{Decompiling input numbers}

Additionally, the software provides a secondary means of use, through
the \tt{-n} flag which stands for \emph{number} so that the argument
immediately following the flag.

Notice in Listing.~\ref{listing:parsing-numbers-from-input-arguments}
that it does not matter whether or not the number is in base 10 or
base 16 (as long as numbers in base 16 are prefixed by \tt{0x}).

Lastly, note that the software handles partially valid instructions,
i.e. instructions that may be correctly identified but validates some
condition. For instance, the number \tt{0x00012122} corresponds to the
\tt{sub} instruction. The number decomposes into bitfields according
to the format of the instruction (viz. the R-format), and the opcode
and funct field hold the correct values to identify the instruction as
a \tt{sub} instruction \emph{but} the \tt{shamt} field is not set to
4, which it has to be according to the instruction specification.

When an instruction is partially legal the violating fields are outputted
on the following line, see Listing.~\ref{listing:partially-valid-instructions}.


\begin{landscape}
\section{Listings}

\begin{lstlisting}[style=plain,
    basicstyle=\footnotesize,
    caption=Tabularized output,
    label=listing:tabularized-output,
][htpb]
java -jar build/libs/kilobyte-all-1.0-SNAPSHOT.jar sample-program.hex sample-mingled-bases.hex sample-decimal.hex
Machine Code  |  Format  |  Decomposition     |  Decomposition hexadecimal  |  Source                    |  Errors
0x23bdfff8    |  I       |  [8 29 29 65528]   |  [8 0x1d 0x1d 0xfff8]       |  addi $sp, $sp, 65528      |  
0xafbf0004    |  I       |  [43 29 31 4]      |  [0x2b 0x1d 0x1f 4]         |  sw $ra, 4($sp)            |  
0xafa40000    |  I       |  [43 29 4 0]       |  [0x2b 0x1d 4 0]            |  sw $a0, 0($sp)            |  
0x28880001    |  I       |  [10 4 8 1]        |  [0xa 4 8 1]                |  slti $t0, $a0, 1          |  
0x11000003    |  I       |  [4 8 0 3]         |  [4 8 0 3]                  |  beq $t0, $zero, 3         |  
0x20020001    |  I       |  [8 0 2 1]         |  [8 0 2 1]                  |  addi $v0, $zero, 1        |  
0x23bd0008    |  I       |  [8 29 29 8]       |  [8 0x1d 0x1d 8]            |  addi $sp, $sp, 8          |  
0x03e00008    |  R       |  [0 31 0 0 0 8]    |  [0 0x1f 0 0 0 8]           |  jr $ra                    |  
0x2084ffff    |  I       |  [8 4 4 65535]     |  [8 4 4 0xffff]             |  addi $a0, $a0, 65535      |  
0x0c100000    |  J       |  [3 0]             |  [3 0]                      |  jal 1048576               |  
0x8fa40000    |  I       |  [35 29 4 0]       |  [0x23 0x1d 4 0]            |  lw $a0, 0($sp)            |  
0x8fbf0004    |  I       |  [35 29 31 4]      |  [0x23 0x1d 0x1f 4]         |  lw $ra, 4($sp)            |  
0x23bd0008    |  I       |  [8 29 29 8]       |  [8 0x1d 0x1d 8]            |  addi $sp, $sp, 8          |  
0x70821002    |  R       |  [28 4 2 2 0 2]    |  [0x1c 4 2 2 0 2]           |  mul $v0, $a0, $v0         |  
0x03e00008    |  R       |  [0 31 0 0 0 8]    |  [0 0x1f 0 0 0 8]           |  jr $ra                    |  
0x00012122    |  R       |  [0 0 1 4 4 34]    |  [0 0 1 4 4 0x22]           |  sub $a0, $zero, $at       |   error(s)=["Expected shamt to be zero. Got 4"]
0x23bdfff8    |  I       |  [8 29 29 65528]   |  [8 0x1d 0x1d 0xfff8]       |  addi $sp, $sp, 65528      |  
0xafbf0004    |  I       |  [43 29 31 4]      |  [0x2b 0x1d 0x1f 4]         |  sw $ra, 4($sp)            |  
0x23bdfff8    |  I       |  [8 29 29 65528]   |  [8 0x1d 0x1d 0xfff8]       |  addi $sp, $sp, 65528      | 
\end{lstlisting}

\begin{figure}
\begin{lstlisting}[style=plain,
    basicstyle=\small,
    caption=Parsing numbers from input arguments using the \tt{-n} flag,
    label=listing:parsing-numbers-from-input-arguments]
$ java -jar build/libs/kilobyte-all-1.0-SNAPSHOT.jar -n 599654392 -n 0xafbf0004 --suppress
0x23bdfff8    |  I       |  [8 29 29 65528]   |  [8 0x1d 0x1d 0xfff8]       |  addi $sp, $sp, 65528      |  
0xafbf0004    |  I       |  [43 29 31 4]      |  [0x2b 0x1d 0x1f 4]         |  sw $ra, 4($sp)            |
\end{lstlisting}
\end{figure}

\begin{lstlisting}[basicstyle=\small,
    style=plain,
    caption=Error print-outs for partially valid instructions,
    label=listing:partially-valid-instructions,
  backgroundcolor=\color{mintedbackground}]
$  java -jar build/libs/kilobyte-all-1.0-SNAPSHOT.jar -n 0x00012122 --suppress
0x00012122    |  R       |  [0 0 1 4 4 34]    |  [0 0 1 4 4 0x22]           |  sub $a0, $zero, $at       |   error(s)=["Expected shamt to be zero. Got 4"]
\end{lstlisting}
\end{landscape}
